\documentclass{amsart}
\renewcommand{\baselinestretch}{1.1}
%\setlength{\textwidth}{6.0in} \setlength{\oddsidemargin}{0.25in}
%\setlength{\evensidemargin}{0.25in}
%\renewcommand{\arraystretch}{.6}
\parskip 2mm

\usepackage{amsthm}
\usepackage{amsmath}
\usepackage{amsfonts}
\usepackage{amssymb}
\usepackage{fullpage}

%\setlength{\oddsidemargin}{.5in} \setlength{\evensidemargin}{.5in}
%\setlength{\textwidth}{6.0in} \theoremstyle{plain}
%\setlength{\topmargin}{-0.5in}\setlength{\textheight}{9.5in}

\newtheorem{theirtheorem}{Theorem}
\newtheorem{theirproposition}{Proposition}
\renewcommand{\thetheirtheorem}{\Alph{theirtheorem}}
\renewcommand{\thetheirproposition}{\Alph{theirproposition}}


\theoremstyle{plain}
\newtheorem{theorem}{\textbf{Theorem}}[section]
\newtheorem{lemma}[theorem]{\textbf{Lemma}}
\newtheorem*{lemma*}{\textbf{Lemma}}
\newtheorem{corollary}[theorem]{\textbf{Corollary}}
\newtheorem{proposition}[theorem]{\textbf{Proposition}}
\newtheorem{claim}{\textbf{Claim}}
\newtheorem{conjecture}{\textbf{Conjecture}}[section]

\newcommand{\be}{\begin{equation}}
\newcommand{\ee}{\end{equation}}
\newcommand{\Summ}[1]{\underset{#1}{\sum}}
\newcommand{\sti}[2]{\left\{\begin{array}{c} #1 \\ #2 \end{array}\right\}}

\newcommand{\diam}{\emph{diam}}
\newcommand{\conv}{\mbox{Conv}}
\newcommand{\C}{\mathcal{C}}
\newcommand{\R}{\mathbb{R}}
\newcommand{\Z}{\mathbb{Z}}
\newcommand{\N}{\mathbb{N}}
\newcommand{\F}{\mathbb{F}}

\newcommand{\B}{\mathcal{B}}
\newcommand{\A}{\mathcal{A}}
\newcommand{\G}{\mathcal{G}}
\newcommand{\D}{\mathcal{D}}

\newcommand{\ov}[1]{\overline{#1}}
\newcommand{\ber}{\begin{eqnarray}}
\newcommand{\eer}{\end{eqnarray}}
\newcommand{\nn}{\nonumber}

\def\st{2}

\thispagestyle{empty}
\begin{document}
{\Large Combinatorics -- Spring 2020}

{\Large Chapter 3 -- The Probabilistic Method: The First Moment Method}

\vspace{0.5cm}

 \hrule

\vspace{0.5cm}

\begin{enumerate}


\item[\textbf{Problem 1:}]
    A \emph{tournament} $T$ is an orientation of the edges of the complete graph on $n$ vertices.
    A tournament has property $S_k$ if for every set $K$ of $k$ vertices there is $x \in V(T)$ such that $(c,y)$ is a directed edge of $T$ for each $y \in K$.
    Show that if,
    \begin{equation*}
        \binom{n}{k}\left(1 - 2^{-k}\right)^{n-k} < 1
    \end{equation*}
    then there is a tournament on $n$ vertices that has the property $S_k$.
        Deduce that this is the case for $n \geq k^2 2^k \text{ln} 2(1 + o(1))$.
\end{enumerate}
%
%
\paragraph{\textbf{Solution (by Carlos Segarra):}} 

bla bla

\end{document}
