\documentclass{amsart}
\renewcommand{\baselinestretch}{1.1}
%\setlength{\textwidth}{6.0in} \setlength{\oddsidemargin}{0.25in}
%\setlength{\evensidemargin}{0.25in}
%\renewcommand{\arraystretch}{.6}
\parskip 2mm

\usepackage{amsthm}
\usepackage{amsmath}
\usepackage{amsfonts}
\usepackage{amssymb}
\usepackage{fullpage}
\usepackage{mathtools}

%\setlength{\oddsidemargin}{.5in} \setlength{\evensidemargin}{.5in}
%\setlength{\textwidth}{6.0in} \theoremstyle{plain}
%\setlength{\topmargin}{-0.5in}\setlength{\textheight}{9.5in}

\newtheorem{theirtheorem}{Theorem}
\newtheorem{theirproposition}{Proposition}
\renewcommand{\thetheirtheorem}{\Alph{theirtheorem}}
\renewcommand{\thetheirproposition}{\Alph{theirproposition}}


\theoremstyle{plain}
\newtheorem{theorem}{\textbf{Theorem}}[section]
\newtheorem{lemma}[theorem]{\textbf{Lemma}}
\newtheorem*{lemma*}{\textbf{Lemma}}
\newtheorem{corollary}[theorem]{\textbf{Corollary}}
\newtheorem{proposition}[theorem]{\textbf{Proposition}}
\newtheorem{claim}{\textbf{Claim}}
\newtheorem{conjecture}{\textbf{Conjecture}}[section]

\newcommand{\be}{\begin{equation}}
\newcommand{\ee}{\end{equation}}
\newcommand{\Summ}[1]{\underset{#1}{\sum}}
\newcommand{\sti}[2]{\left\{\begin{array}{c} #1 \\ #2 \end{array}\right\}}

\newcommand{\diam}{\emph{diam}}
\newcommand{\conv}{\mbox{Conv}}
\newcommand{\C}{\mathcal{C}}
\newcommand{\R}{\mathbb{R}}
\newcommand{\Z}{\mathbb{Z}}
\newcommand{\N}{\mathbb{N}}
\newcommand{\F}{\mathbb{F}}

\newcommand{\B}{\mathcal{B}}
\newcommand{\A}{\mathcal{A}}
\newcommand{\G}{\mathcal{G}}
\newcommand{\D}{\mathcal{D}}

\newcommand{\ov}[1]{\overline{#1}}
\newcommand{\ber}{\begin{eqnarray}}
\newcommand{\eer}{\end{eqnarray}}
\newcommand{\nn}{\nonumber}

\def\st{2}

\thispagestyle{empty}
\begin{document}
{\Large Combinatorics -- Spring 2020}

{\Large Chapter 3 -- The Probabilistic Method: The First Moment Method}

\vspace{0.5cm}

 \hrule

\vspace{0.5cm}

\begin{enumerate}


\item[\textbf{Problem 1:}]
    A \emph{tournament} $T$ is an orientation of the edges of the complete graph on $n$ vertices.
    A tournament has property $S_k$ if for every set $K$ of $k$ vertices there is $x \in V(T)$ such that $(x,y)$ is a directed edge of $T$ for each $y \in K$.
    Show that if,
    \begin{equation*}
        \binom{n}{k}\left(1 - 2^{-k}\right)^{n-k} < 1
    \end{equation*}
    then there is a tournament on $n$ vertices that has the property $S_k$.
        Deduce that this is the case for $n \geq k^2 2^k \text{ln} 2(1 + o(1))$.
\end{enumerate}
%
%
\paragraph{\textbf{Solution (by Carlos Segarra):}} 
let $T$ be a random tournament, this is, a random orientation of the edges of the complete graph on $n$ vertices $K_n$.
To construct it, we can think of it as a random coloring.
First, we pick an arbitrary direction for each edge, then with probability $\frac{1}{2}$, we color it with color 0 (and hence, for instance, conserve the chosen direction), or with $1$, and we invert said direction.

Let now $K$ be a subset of $V(T)$ of size $k$, with vertices $\{ y_1, \dots, y_k \}$.
We denote by $X_K$ the indicator variable that $K$ \emph{does not} fulfil the $S_k$ property.
This is, there does not exist $x \in V(T)$ such that $\{xy_1, \dots, xy_k \} \subset E(T)$.
We now compute the probability of such event.
As usually done, we will start with the probability that such event does indeed happen, once we fix $x \in V(T)$.
Observe that this means that each edge egressing $x$ to a vertex in $K$ is indeed picked (or colored) in this direction.
Then,
\begin{equation*}
    \begin{split}
        Pr\left(\{xy_1, \dots, xy_k \} \in E(T), x \text{ fixed }\right) & = \frac{1}{2}^k = 2^{-k} \Rightarrow Pr\left(\{xy_1, \dots, xy_k \} \not\in E(T), x \text{ fixed }\right) = 1 - 2^{-k}.
    \end{split}
\end{equation*}
And as the choice of $x$ is independent from the probability,
\begin{equation*}
    \begin{split}
        Pr(X_K = 1) = Pr\left(\{xy_1, \dots, xy_k \} \not\in E(T), \forall x \in V(T)\backslash K \right) = \left(1 - 2^{-k}\right)^{n-k}.
    \end{split}
\end{equation*}
Observe that we implicitly assume that $x \not\in K$.
It is clear that if $x$ is in $T$ the property can not hold as we use simple graphs, hence without loops.

As this probability is independent from the choice of $K \subset V(T)$, we can finally compute the expected number of subsets that \emph{do not} fulfil the $S_k$ property (which we denote with the random variable $X$),
\begin{equation*}
    \begin{split}
        \mathbb{E}(X) = \sum_{K \in \binom{[n]}{k}} \mathbb{E}(X_K) = \sum_{K \in \binom{[n]}{k}} Pr(X_K = 1) = \binom{n}{k} (1 - 2^{-k})^{n-k} \leq 1.
    \end{split}
\end{equation*}
Lastly, applying Markov's inequality,
\begin{equation*}
    Pr(X \geq 1) \leq \mathbb{E}(X) < 1 \Rightarrow Pr(X = 0) > 0,
\end{equation*}
and there exists a coloring such that no subset \emph{does not} fulfil the $S_K$ property, and hence the statement holds.

For the second part of the statement, we know that the initial hypothesis must hold for such a tournament to exist.
We then need to compute for which values of $n$ this does happen.
Using the usual bounds, $\binom{n}{k} \leq n^k$ and $e^{-x} \simeq 1 - x$, then:
\begin{equation*}
    \begin{split}
        \binom{n}{k}\left(1 - 2^{-k}\right)^{n-k} \leq n^k e^{-2^{-k}(n - k)} \overset{n \gg k}{\simeq} n^k e^{-2^kn}.
    \end{split}
\end{equation*}
As the application is increasing in $n$, if we prove the lower bound in $n$ for a looser bound, then it will also hold for the tighter.
Hence, there exists a tournament with the $S_k$ property if $n$ satisfies:
\begin{equation*}
    \begin{split}
        n^k e^{-2^kn} < 1 \Leftrightarrow n^k < e^{2^kn} \Leftrightarrow 2^k k \text{ln} (n) < n.
    \end{split}
\end{equation*}
Let $\alpha \coloneqq 2^k k$, then we want to find the smallest value of $n$ for which $\alpha \text{ln}(n) < n$.
Following the indication from the statement, we claim that if $n = \alpha \ln\alpha (1 + o(1))$ then the inequality holds (note that this bound is slightly better than the one in the statement as $k \geq 1$).
We see,
\begin{equation*}
    \begin{split}
        \alpha \ln (n) = \alpha \ln \left( \alpha \ln(\alpha) (1 + o(1)) \right) = \alpha \ln (\alpha) + \alpha \ln ( \ln (\alpha) (1 + o(1)) \leq \alpha \ln (\alpha) (1 + o(1)) = n
    \end{split}
\end{equation*}
what proves the result. \qed


\end{document}
